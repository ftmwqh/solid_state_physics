\documentclass{article}
\usepackage{ctex}
\usepackage{graphicx}
\usepackage{amsmath}
\usepackage{indentfirst}
\usepackage{titlesec}
\usepackage{setspace}
\usepackage{subfigure}
\usepackage{caption}
\usepackage{float}
\usepackage{booktabs}
\usepackage{geometry}
\usepackage{multirow}
\usepackage{hyperref}
\hypersetup{
	colorlinks=true,
	linkcolor=blue,
	filecolor=magenta,      
	urlcolor=cyan,
	pdftitle={Overleaf Example},
	pdfpagemode=FullScreen,
}
\geometry{left=1.2cm,right=1.2cm,top=2cm,bottom=2cm}
\title{\songti \zihao{2}\bfseries 二维平面的态密度}
\titleformat*{\section}{\songti\zihao{4}\bfseries}
\titleformat*{\subsection}{\songti\zihao{5}\bfseries}
\renewcommand\thesection{\arabic{section}}
\author{王启骅 PB20020580}
\begin{document}
	\maketitle
在二维平面上,k空间中2$\rho(\boldsymbol{k})\times$能量在$ E-E+dE $的等能线之间的面积即为E到E+dE的能态数,其中
\begin{equation}
	\rho(\boldsymbol{k})=\frac{S}{(2\pi)^2}
\end{equation}
则
\begin{equation}
	dZ=2\frac{S}{(2\pi)^2}\int_{E=const}dldk_{\perp}
\end{equation}
带入
\begin{equation}
	dE=|\nabla_{k}E|\cdot dk_{\perp}
\end{equation}
得到
\begin{equation}
	N(E)=\frac{1}{S}\dfrac{dZ}{dE}=\frac{1}{2\pi^2}\int_{E=const}\frac{dl}{|\nabla_{k}E(k)|}
\end{equation}


根据电子抛物线色散关系
\begin{equation}
	E(\boldsymbol{k})=\frac{\hbar^2k^2}{2m}
\end{equation}
\begin{equation}
	|\nabla_{k}E(k)|=\frac{\hbar^2}{m}k
\end{equation}
带入积分得到
\begin{equation}
	N(E)=\frac{1}{2\pi^2}\frac{m}{\hbar2k}2\pi k=\frac{m}{\pi\hbar^2}
\end{equation}
可以比对与石墨烯的态密度,得到二者相等。


但是在石墨烯位于布里渊区格点附近的低能电子,具有线性色散关系
\begin{equation}
	E=\hbar v_F \sqrt{k_x^2+k_y^2}
\end{equation}
其中
\begin{equation}
	v_F\sim10^6m/s
\end{equation}
为费米速度,则
\begin{equation}
	|\nabla_{k}E|=|\hbar v_F \frac{k_x\hat{x}+k_y\hat{y}}{k}|=\hbar v_F
\end{equation}
得到
\begin{equation}
	N(E)=\frac{1}{2\pi^2}\int_{E=const}\frac{dl}{|\nabla_{k}E(k)|}=\frac{E}{\pi\hbar^2 v_F^2}
\end{equation}


可见对于二维下线性色散,N$\propto\sqrt{E}$,由此可见与抛物线形式色散为常数的区别。

\end{document}